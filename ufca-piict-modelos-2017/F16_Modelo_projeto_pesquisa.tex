%
% `F15_Modelo_projeto_pesquisa.tex'.
%
% Este é um template em LaTeX para submissão de propostas para o Programa
% Institucional de Iniciação Científica e Tecnológica (PIICT) da Universidade
% Federal do Cariri (UFCA).
%
\documentclass[a4paper,12pt]{article}
\usepackage[utf8]{inputenc}
\usepackage[brazil]{babel}
\usepackage[lmargin=3cm,rmargin=3cm,tmargin=2.5cm,bmargin=2.5cm]{geometry}
\usepackage[table]{xcolor}
\usepackage{longtable}
\usepackage{soulutf8}
\usepackage[hyphens]{url}
\usepackage{titlesec}
\titleformat*{\section}{\large}
\makeatletter
\def\@seccntformat#1{\csname the#1\endcsname\ }
\makeatother
\def\BibTeX{{\rm B\kern-.05em{\sc i\kern-.025em b}\kern-.08em
    T\kern-.1667em\lower.7ex\hbox{E}\kern-.125emX}}
%
% Adicione seus pacotes logo abaixo.
%


\begin{document}
\noindent{\fontsize{16pt}{19.2pt}\selectfont\bfseries Identificação da proposta
\hl{[Aviso: o projeto deve ser apresentado em, no máximo, 15 paginas]}}
\vspace*{0.85cm}\par
\noindent{\fontsize{14pt}{16.8pt}\selectfont\bfseries Título}
\vspace*{0.35cm}\par
\noindent Lorem ipsum dolor sit amet, consectetur adipiscing eli
\textcolor{red}{\bf (máximo de 200 caracteres)}
\vspace*{0.35cm}\par
\noindent{\fontsize{14pt}{16.8pt}\selectfont\bfseries Resumo}
\vspace*{0.35cm}\par
\noindent Donec sem est, dictum ut malesuada vel, rutrum quis purus. Phasellus euismod
semper pharetra. Aenean sed nibh velit. Sed pulvinar felis pulvinar
sollicitudin laoreet. Aliquam eget est sed ex faucibus pharetra. Proin interdum
sit amet quam sit amet pellentesque. Nullam purus sem, eleifend vitae lacus
quis, fringilla facilisis elit. Sed pretium leo eu urna euismod dictum. Nulla
vulputate quam dolor, sit amet imperdiet nisi pellentesque sit amet. Sed
tincidunt enim pellentesque metus convallis eleifend. Mauris ultricies, sem ut
ornare bibendum, leo risus rutrum dolor, non ornare sem neque euismod neque.
Proin tincidunt venenatis ultrices. Donec pulvinar urna odio, ac porta ex
luctus eget. Ut quis consectetur arcu, eget mollis ex. Phasellus euismod semper
pharetra. Aenean sed nibh velit. Sed pulvinar felis pulvinar sollicitudin
laoreet. Aliquam eget est sed ex faucibus pharetra. Ut quis consectetur arcu,
eget mollis ex.  \textcolor{red}{\bf (de 100 a 300 palavras)}

\vspace*{0.35cm}\par
\noindent{\fontsize{14pt}{16.8pt}\selectfont\bfseries Palavras-chave}
\vspace*{0.35cm}\par
\noindent Lorem ipsum; dolor sit; amet. \textcolor{red}{\bf(até quatro palavras-chave)}

\vspace*{0.35cm}\par
\noindent{\fontsize{14pt}{16.8pt}\selectfont\bfseries Área do conhecimento predominante}
\vspace*{0.35cm}\par
\noindent Ciências biológicas I \textcolor{red}{\bf (preencher segundo a tabela de áreas da Capes:
\url{http://www.capes.gov.br/avaliacao/instrumentos-de-apoio/tabela-de-areas-do-conhecimento-avaliacao})}

\section{Introdução}

Descrever objetivamente, com fundamentação teórica, o problema em estudo.

\section{Justificativa}

Deve conter a articulação de argumentos de forma a demonstrar a relevância e
originalidade do tema em estudo no contexto da área inserida e sua importância
específica para o avanço do conhecimento.

\section{Referencial teórico}

Destina-se a apresentar as leituras e fundamentos teóricos prévios que embasam
a proposta de pesquisa.

\section{Objetivos}

Os objetivos indicam o que é pretendido com o desenvolvimento da pesquisa.
Estes são comumente divididos em: objetivo geral e objetivos específicos;
embora isto não seja obrigatório. O objetivo geral, geralmente único, é uma
meta global que norteará toda a pesquisa. Já os objetivos específicos são
desdobramentos do objetivo geral.

\section{Metodologia}

A seção de Metodologia consiste na descrição formal dos métodos e técnicas a
serem utilizados na pesquisa.

\section{Modalidades de bolsa}

Nesta seção, deve-se especificar as quantidades de bolsistas pretendidas,
indicando suas respectivas modalidades (PIBIC, PIBITI ou PIBIC Ensino Médio).

\section{Cronograma de atividades}

O proponente deve descrever os objetivos e o tema do trabalho de cada bolsista,
elencando com clareza todas as ações a serem realizadas, distribuindo-as no
espaço de tempo disponível para a realização do projeto, conforme exemplo
abaixo.

\begin{center}
\renewcommand{\arraystretch}{1.2}
\begin{longtable}{|cp{0.85\textwidth}|}
\hline
\rowcolor{lightgray}
\multicolumn{2}{|c|}{\textbf{BOLSISTA 1}}\\
\hline
\multicolumn{2}{|l|}{\textbf{MODALIDADE:} PIBIC}\\
\hline
\multicolumn{2}{|l|}{\textbf{OBJETIVOS:}}\\
\multicolumn{2}{|m{0.95\textwidth}|}{%
Sed pulvinar felis pulvinar sollicitudin laoreet. Aliquam eget est sed ex
faucibus pharetra. Proin interdum sit amet quam sit amet pellentesque. Nullam
purus sem, eleifend vitae lacus quis, fringilla facilisis elit. Sed pretium leo
eu urna euismod dictum. Nulla vulputate quam dolor, sit amet imperdiet nisi
pellentesque sit amet.}\\
\hline
\multicolumn{1}{|c|}{\textbf{MÊS}} & \multicolumn{1}{c|}{\textbf{ATIVIDADE}}\\
\hline
\multicolumn{1}{|c|}{1} & 
\multicolumn{1}{m{0.86\textwidth}|}{%
Sed pulvinar felis pulvinar sollicitudin laoreet. Aliquam eget est sed ex
faucibus pharetra. Proin interdum sit amet quam sit amet pellentesque. Nullam
purus sem, eleifend vitae lacus quis, fringilla facilisis elit. Sed pretium leo
eu urna euismod dictum. Nulla vulputate quam dolor, sit amet imperdiet nisi}\\
\hline
\multicolumn{1}{|c|}{2} & \\
\hline
\multicolumn{1}{|c|}{3} & \\
\hline
\multicolumn{1}{|c|}{4} & \\
\hline
\multicolumn{1}{|c|}{5} & \\
\hline
\multicolumn{1}{|c|}{6} & \\
\hline
\multicolumn{1}{|c|}{7} & \\
\hline
\multicolumn{1}{|c|}{8} & \\
\hline
\multicolumn{1}{|c|}{9} & \\
\hline
\multicolumn{1}{|c|}{10} & \\
\hline
\multicolumn{1}{|c|}{11} & \\
\hline
\multicolumn{1}{|c|}{12} & \\
\hline
\end{longtable}
\end{center}

\section{Resultados esperados e impactos}

Descreve-se aqui os resultados que pretende-se obter de modo concreto e os
possíveis impactos resultantes da pesquisa no âmbito institucional, estadual,
regional, nacional ou na área de conhecimento do projeto.

\section*{Referências}

Esta seção têm como finalidade fornecer um conjunto de indicações precisas que
permitam ao leitor identificar todas as fontes bibliográficas para elaboração
do projeto. O autor deve utilizar as normas vigentes da ABNT. Para os projetos
submetidos em \LaTeX, recomenda-se usar o \BibTeX\ por intermédio do abn\TeX2
(\url{http://www.abntex.net.br/}).
%\bibliographystyle{abntex2-num}
%\bibliography{projeto-piict-2017}

\end{document}
